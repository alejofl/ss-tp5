% Preamble
\documentclass{beamer}
\usepackage[spanish]{babel}

% Packages
\usepackage{amsmath}
\usepackage[utf8]{inputenc}
\usepackage[T1]{fontenc}
\usepackage{graphicx}
\usepackage{algorithmicx}
\usepackage{algpseudocode}
\usepackage{caption}
\usepackage{courier}

\DeclareMathOperator{\atantwo}{atan2}

\captionsetup{justification=centering, font={scriptsize}, skip=0pt}

% Set bold vectors to satisfy requirements
\renewcommand\vec[1]{\ifstrequal{#1}{0}{\ensuremath{\mathbf{0}}}{\ensuremath{\boldsymbol{#1}}}}

\usetheme[compress]{Berlin}
\usecolortheme{wolverine}
\setbeamertemplate{page number in head/foot}[framenumber]
\setbeamercolor{institute in head/foot}{parent=palette primary}

\title[Dinámica peatonal]{Dinámica peatonal}
\subtitle{72.25 - Simulación de Sistemas}
\author[Flores Lucey, Llanos]{Alejo Flores Lucey\inst{1} \and Nehuén Gabriel Llanos\inst{2}}
\institute[Instituto Tecnológico de Buenos Aires]
{
    \inst{1}
    \href{mailto:afloreslucey@itba.edu.ar}{afloreslucey@itba.edu.ar}\\
    Legajo 62622
    \and
    \inst{2}
    \href{mailto:nllanos@itba.edu.ar}{nllanos@itba.edu.ar}\\
    Legajo 62511
}
\date{2024 1C | Grupo Nº3}
\titlegraphic{\includegraphics[height=0.5cm]{./itba}}

\makeatletter
\beamer@theme@subsectionfalse
\makeatother

\AtBeginSection[]{
    \begin{frame}
        \begin{beamercolorbox}[sep=8pt,center]{title}
            \usebeamerfont{title}\insertsection
        \end{beamercolorbox}
    \end{frame}
}

\begin{document}

    \begin{frame}
        \titlepage
    \end{frame}

    \section{Introducción}

        \begin{frame}{Introducción}
            \begin{itemize}
                \item Dinámica de interacción entre personas en movimiento.
                \item Simulación de un jugador virtual que sigue a la pelota durante un partido de fútbol en dos dimensiones.
            \end{itemize}
            \begin{minipage}[t]{0.5\textwidth}
                \begin{itemize}
                    \item Sistema real:
                    \begin{itemize}
                        \item Procesos de evacuación.
                        \item Predicción del flujo de personas.
                    \end{itemize}
                \end{itemize}
            \end{minipage}
            \hfill
            \begin{minipage}[t]{0.45\textwidth}
                \begin{figure}[H]
                    \centering
                    \includegraphics[width=\linewidth]{./sistema_real}
                    \label{fig:sistema_real}
                \end{figure}
            \end{minipage}
        \end{frame}

        \subsection{Fundamentos}

            \begin{frame}{Fundamentos: Social Force Model}
                \begin{itemize}
                    \item Sean $N$ peatones en un espacio bidimensional:
                        \begin{equation*}
                            m_i \cdot \vec{a}_i = \vec{F}_{i}^{(granular)} + \vec{F}_{i}^{(social)} + \vec{F}_{i}^{(deseo)}
                        \end{equation*}
                    \item Distancia entre peatones:
                        \begin{equation*}
                            \varepsilon_{ij} = d_{ij} - (R_i + R_j)
                        \end{equation*}
                    \item Fuerza granular:
                        \begin{equation*}
                            \vec{F}_{i}^{(granular)} = \sum_{j \neq i}^{N} -k_n \cdot g(\varepsilon_{ij}) \cdot \vec{e}_{ij}^{n}\ ;\ g(w) = \begin{cases} w & \text{si } w < 0 \\ 0 & \text{si } w \geq 0 \end{cases}
                        \end{equation*}
                \end{itemize}
                \begin{minipage}[t]{0.49\textwidth}
                    \begin{itemize}
                        \item Fuerza social:
                        \begin{equation*}
                            \vec{F}_{i}^{(social)} = \sum_{j \neq i}^{N} A \cdot e^{\frac{-\varepsilon_{ij}}{B}} \cdot \vec{e}_{ij}^{n}
                        \end{equation*}
                    \end{itemize}
                \end{minipage}
                \hfill
                \begin{minipage}[t]{0.49\textwidth}
                    \begin{itemize}
                        \item Fuerza de deseo:
                        \begin{equation*}
                            \vec{F}_{i}^{(deseo)} = m_i \cdot \frac{\vec{v}_i^{deseo} - \vec{v}_i}{\tau}
                        \end{equation*}
                    \end{itemize}
                \end{minipage}
            \end{frame}

    \section{Implementación}

        \subsection{Arquitectura}

            \begin{frame}{Diagrama UML}
                \vspace*{-0.2cm}
                \begin{figure}[htbp]
                    \centering
                    \includegraphics[width=\textwidth]{./architecture}
                    \label{fig:architecture}
                \end{figure}
            \end{frame}

        \subsection{Algoritmo}

            \begin{frame}{Pseudocódigo del algoritmo implementado}{}
                \begin{algorithmic}[1]
                    \ttfamily \scriptsize
                    \State Create output file
                    \State accX $\gets$ \Call{function}{ball, players}
                    \State accY $\gets$ \Call{function}{ball, players}
                    \State integrators $\gets$ [accX, accY]
                    \State $t$ $\gets$ startTime
                    \State $i$ $\gets$ 0
                    \While{$t$ $<$ endTime}
                        \State crazyGuyX, crazyGuyY $\gets$ \Call{integrators.next}{}()
                        \State \Call{UpdateStateVariables}{crazyGuyX, crazyGuyY}
                        \If{$i \equiv 0(10)$}
                            \State \Call{WriteOutput}{crazyGuyX, crazyGuyY}
                        \EndIf
                        \State $t \gets t + dt$
                        \State $i \gets i + 1$
                    \EndWhile
                \end{algorithmic}
            \end{frame}

    \section{Simulaciones}

        \subsection{Parámetros de entrada}

            \begin{frame}{Parámetros de entrada}
                \begin{itemize}
                    \item Parámetros de entrada fijos:
                    \begin{itemize}
                        \item $d_{pelota-loco}$: \alert{$10$ m}
                        \item $R_{jugadores}$: \alert{$0.3$ m}
                        \item $m_{jugadores}$: \alert{$80$ kg}
                        \item $r_{jugador_i}$: \alert{[$(x, y)$, ...]}
                    \end{itemize}
                    \item Parámetros de entrada variables:
                    \begin{itemize}
                        \item $v_d$: \alert{${0.1; 1; 2; 3; 4; ...; 13}$ m/s}
                        \item $\tau$: \alert{$0.1; 0.2; ...; 1$ s}
                    \end{itemize}
                \end{itemize}
                \begin{block}{Dato curioso}
                    Como método de integración se utilizó \textbf{Verlet 2.0}, presentado en el TP anterior.
                \end{block}
            \end{frame}

        \subsection{Cálculo de variables importantes}

            \begin{frame}{Elección del periodo de simulación}
                \begin{itemize}
                    \item Se busca un periodo \alert{$[t_i ; t_f]$} tal que:
                    \begin{itemize}
                        \item $t_i > 120$ s $\land\ t_f < 6000$ s
                        \item Se mantengan los mismos jugadores en la cancha.
                        \item $x_{pelota} \ne$ NaN $\land\ y_{pelota} \ne$ NaN $\forall t \in [t_i ; t_f]$
                    \end{itemize}
                    \item Se encuentra que:
                    \begin{itemize}
                        \item $\max\{[t_i ; t_f] : \text{mismos jugadores en la cancha}\} = [0.04 ; 3519.40]$
                        \item $\max\{[t_i ; t_f] : x_{pelota} \ne \text{NaN} \land\ y_{pelota} \ne \text{NaN}\} = [57.24 ; 177.72]$
                    \end{itemize}
                \end{itemize}
                \begin{beamercolorbox}[sep=5pt,center]{block body}
                    \centering
                    $\therefore [t_i ; t_f] = [57.24 ; 177.72] \Rightarrow D = 120.48$ s
                \end{beamercolorbox}
            \end{frame}

            \begin{frame}{Cálculo de la velocidad de deseo en cada instante}
                \begin{itemize}
                    \item Tomamos el módulo de la velocidad de deseo de parámetro de entrada.
                    \item Obtenemos la velocidad de deseo en cada instante $t$:
                        \begin{equation*}
                            \hat{e_n} = \frac{\vec{r}(t)_{pelota} - \vec{r}(t)_{loco}}{||\vec{r}(t)_{pelota} - \vec{r}(t)_{loco}||}= (e_n \hat{\imath}, e_n \hat{\jmath})
                        \end{equation*}
                        \begin{equation*}
                            \vec{v_d} = |v_d| \cdot \hat{e_n}
                        \end{equation*}
                \end{itemize}
            \end{frame}

        \subsection{Observables}

            \begin{frame}{Distancia entre el jugador virtual y la pelota}
                \begin{equation*}
                    d = \sqrt{(x_{pelota} - x_{loco})^2 + (y_{pelota} - y_{loco})^2} - R_{loco}
                \end{equation*}
                \begin{itemize}
                    \item Se busca minimizar la distancia entre el jugador virtual y la pelota, variando $v_d$ y $\tau$.
                    \item $d$ no presenta un comportamiento gaussiano $\Rightarrow$ Cuando se toman promedios, el error se calcula como:
                    \begin{equation*}
                        E = \frac{\sigma}{\sqrt{N}}\ ;\ \sigma = \text{desvío estándar}\ ;\ N = \text{cantidad de datos}
                    \end{equation*}
                \end{itemize}
            \end{frame}

            \begin{frame}{Módulo de la velocidad de distintos jugadores}
                \begin{itemize}
                    \item Se define la función de densidad de probabilidad de la velocidad de un jugador como:
                    \begin{equation*}
                        f(v(t)) \text{ donde } v(t) = \frac{|\vec{r}(t) - \vec{r}(t - \Delta t)|}{\Delta t}\ ;\ \Delta t = 0.04 \text{s}
                    \end{equation*}
                    \item Se divide el intervalo de velocidades en $\Delta x = 0,04$
                    \item Se eliminan las velocidades que superen los $5,5$m/s.
                    \item Se estudia la función de densidad de probabilidad de la velocidad del jugador virtual y los jugadores 1, 6 y 11.
                \end{itemize}
            \end{frame}

            \begin{frame}{Número de visitas a zonas de la cancha por minuto ($\Psi$)}
                \begin{itemize}
                    \item Se divide la cancha en zonas de $2 \times 2 \text{ m}^2$.
                    \begin{equation*}
                        \Psi = \frac{\sum_{i \in \text{jugadores}} \#visitas_i}{D}
                    \end{equation*}
                    \begin{equation*}
                        D = \text{periodo de simulación} = 2.008 \text{ minutos}
                    \end{equation*}
                    \item Se estudia $\Psi$ utilizando un \alert{heatmap}.
                \end{itemize}
            \end{frame}

    \section{Resultados}

        \subsection{Animación del sistema}

            \begin{frame}{Animación del sistema}{}
                \vspace*{-0.3cm}
                \begin{figure}[H!]
                    \includegraphics[width=\textwidth]{./animacion_1}
                    \caption*{Véase la animación completa en \url{TODO}.}
                    \label{fig:futbol_1}
                \end{figure}
                \vspace*{-0.5cm}
                \begin{beamercolorbox}[sep=5pt,center]{block body}
                    \centering
                    \small{$v_d = 5$ m/s ; $\tau = 0.5$ s}
                \end{beamercolorbox}
            \end{frame}

        \subsection{Distancia entre el jugador virtual y la pelota}

            \begin{frame}{Variación de $v_d$}{Animación #1}
                \vspace*{-0.3cm}
                \begin{figure}[H!]
                    \includegraphics[width=\textwidth]{./animacion_2}
                    \caption*{Véase la animación completa en \url{TODO}.}
                    \label{fig:futbol_2}
                \end{figure}
                \vspace*{-0.5cm}
                \begin{beamercolorbox}[sep=5pt,center]{block body}
                    \centering
                    \small{$v_d = 0.1$ m/s ; $\tau = 0.5$ s}
                \end{beamercolorbox}
            \end{frame}

            \begin{frame}{Variación de $v_d$}{Animación #2}
                \vspace*{-0.3cm}
                \begin{figure}[H!]
                    \includegraphics[width=\textwidth]{./animacion_3}
                    \caption*{Véase la animación completa en \url{TODO}.}
                    \label{fig:futbol_3}
                \end{figure}
                \vspace*{-0.5cm}
                \begin{beamercolorbox}[sep=5pt,center]{block body}
                    \centering
                    \small{$v_d = 10$ m/s ; $\tau = 0.5$ s}
                \end{beamercolorbox}
            \end{frame}

            \begin{frame}{Distancia entre el loco y la pelota en función del tiempo}{Variación de $v_d$}
                    \begin{figure}[H!]
                        \includegraphics[width=0.9\textwidth]{./distancia_vs_tiempo_vd}
                        \label{fig:futbol_4}
                    \end{figure}
                    \begin{beamercolorbox}[sep=5pt,center]{block body}
                        \centering
                        \small{$\tau = 0.5$ s}
                    \end{beamercolorbox}
            \end{frame}

            \begin{frame}{Promedio de la distancia loco-pelota en función de $v_d$}{Variación de $v_d$}
                \begin{figure}[H!]
                    \includegraphics[width=0.9\textwidth]{./distancia_vs_vd}
                    \label{fig:futbol_5}
                \end{figure}
                \begin{beamercolorbox}[sep=5pt,center]{block body}
                    \centering
                    \small{$\tau = 0.5$ s}
                \end{beamercolorbox}
            \end{frame}

            \begin{frame}{Variación de $\tau$}{Animación #1}
                \vspace*{-0.3cm}
                \begin{figure}[H!]
                    \includegraphics[width=\textwidth]{./animacion_4}
                    \caption*{Véase la animación completa en \url{TODO}.}
                    \label{fig:futbol_6}
                \end{figure}
                \vspace*{-0.5cm}
                \begin{beamercolorbox}[sep=5pt,center]{block body}
                    \centering
                    \small{$v_d = 5$ m/s ; $\tau = 0.1$ s}
                \end{beamercolorbox}
            \end{frame}

            \begin{frame}{Variación de $\tau$}{Animación #2}
                \vspace*{-0.3cm}
                \begin{figure}[H!]
                    \includegraphics[width=\textwidth]{./animacion_5}
                    \caption*{Véase la animación completa en \url{TODO}.}
                    \label{fig:futbol_7}
                \end{figure}
                \vspace*{-0.5cm}
                \begin{beamercolorbox}[sep=5pt,center]{block body}
                    \centering
                    \small{$v_d = 5$ m/s ; $\tau = 1$ s}
                \end{beamercolorbox}
            \end{frame}

            \begin{frame}{Distancia entre el loco y la pelota en función del tiempo}{Variación de $\tau$}
                \begin{figure}[H!]
                    \includegraphics[width=0.9\textwidth]{./distancia_vs_tiempo_tau}
                    \label{fig:futbol_8}
                \end{figure}
                \begin{beamercolorbox}[sep=5pt,center]{block body}
                    \centering
                    \small{$v_d = 5$ m/s}
                \end{beamercolorbox}
            \end{frame}

            \begin{frame}{Promedio de la distancia loco-pelota en función de $\tau$}{Variación de $\tau$}
                \begin{figure}[H!]
                    \includegraphics[width=0.9\textwidth]{./distancia_vs_tau}
                    \label{fig:futbol_9}
                \end{figure}
                \begin{beamercolorbox}[sep=5pt,center]{block body}
                    \centering
                    \small{$v_d = 5$ m/s}
                \end{beamercolorbox}
            \end{frame}

        \subsection{Módulo de la velocidad de distintos jugadores}

            \begin{frame}{Función de densidad de probabilidad}{Módulo de la velocidad de distintos jugadores}
                \begin{figure}[H!]
                    \includegraphics[width=0.9\textwidth]{./pdf}
                    \label{fig:futbol_10}
                \end{figure}
                \begin{beamercolorbox}[sep=5pt,center]{block body}
                    \centering
                    \small{$v_d = 5$ m/s ; $\tau = 0.5$ s}
                \end{beamercolorbox}
            \end{frame}

        \subsection{Número de visitas a zonas de la cancha por minuto}

            \begin{frame}{Heatmap de todos los jugadores}{Número de visitas a zonas de la cancha por minuto}
                \begin{figure}[H!]
                    \includegraphics[width=0.9\textwidth]{./heatmap_all}
                    \label{fig:futbol_11}
                \end{figure}
                \begin{beamercolorbox}[sep=5pt,center]{block body}
                    \centering
                    \small{Se estudia el intervalo entre $57.24$ s y $177.72$ s}
                \end{beamercolorbox}
            \end{frame}

            \begin{frame}{Heatmap del jugador virtual}{Número de visitas a zonas de la cancha por minuto}
                \begin{figure}[H!]
                    \includegraphics[width=0.9\textwidth]{./heatmap_loco}
                    \label{fig:futbol_12}
                \end{figure}
                \begin{beamercolorbox}[sep=5pt,center]{block body}
                    \centering
                    \small{Se estudia el intervalo entre $57.24$ s y $177.72$ s}
                \end{beamercolorbox}
            \end{frame}

            \begin{frame}{Heatmap del jugador \#1 (lateral derecho)}{Número de visitas a zonas de la cancha por minuto}
                \begin{figure}[H!]
                    \includegraphics[width=0.9\textwidth]{./heatmap_jugador_1}
                    \label{fig:futbol_13}
                \end{figure}
                \begin{beamercolorbox}[sep=5pt,center]{block body}
                    \centering
                    \small{Se estudia el intervalo entre $57.24$ s y $177.72$ s}
                \end{beamercolorbox}
            \end{frame}

            \begin{frame}{Heatmap del jugador \#6 (mediocampista central)}{Número de visitas a zonas de la cancha por minuto}
                \begin{figure}[H!]
                    \includegraphics[width=0.9\textwidth]{./heatmap_jugador_6}
                    \label{fig:futbol_14}
                \end{figure}
                \begin{beamercolorbox}[sep=5pt,center]{block body}
                    \centering
                    \small{Se estudia el intervalo entre $57.24$ s y $177.72$ s}
                \end{beamercolorbox}
            \end{frame}

            \begin{frame}{Heatmap del jugador \#11 (arquero)}{Número de visitas a zonas de la cancha por minuto}
                \begin{figure}[H!]
                    \includegraphics[width=0.9\textwidth]{./heatmap_jugador_11}
                    \label{fig:futbol_15}
                \end{figure}
                \begin{beamercolorbox}[sep=5pt,center]{block body}
                    \centering
                    \small{Se estudia el intervalo entre $57.24$ s y $177.72$ s}
                \end{beamercolorbox}
            \end{frame}

    \section{Conclusiones}

        \begin{frame}{Conclusiones}
            \begin{itemize}
                \item $v_d$ aumenta $\Rightarrow$ $d_{pelota-loco}$ aumenta
                \item $\tau$ aumenta $\Rightarrow$ $d_{pelota-loco}$ disminuye
                \item $\vec{v_{loco}}$ tiende a $\vec{v_d}$
                \item Se observan diferentes comportamientos para diferentes posiciones de los jugadores
            \end{itemize}
        \end{frame}

        \begin{frame}
            \begin{beamercolorbox}[sep=8pt,center]{title}
                \usebeamerfont{title}{¡Muchas gracias!}
            \end{beamercolorbox}
        \end{frame}

\end{document}
